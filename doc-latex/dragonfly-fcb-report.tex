\documentclass[a4paper]{article}

\usepackage{url}
\usepackage{amsmath}
\usepackage{verbatim}   		% Useful for program listings
\usepackage[T1]{fontenc}       	% For Swedish characters ÅÄÖ etc.
\usepackage[utf8]{inputenc}	
% \usepackage[swedish]{babel} % For Swedish hyphenation
\usepackage{fancyvrb}           	% For lists with tabulators
\fvset{tabsize=4}              	 	% Tabulator size
\fvset{fontsize=\small}         	% List font size
\usepackage{graphicx}		% Imports the graphicx package, useful for images
% \usepackage{}

\title{Dragonfly Quadrotor UAV \\ Flight Control Board}
\author{Daniel Stenberg \\ Adam Steineck \\ Nina Khayyami}

\date{September, 2015}         		% Today's date if not specified

\begin{document}                	% Start of document

\maketitle                      		% Prints the title defined above with \title, \author and \date

\begin{center}   
\vspace{64pt}                  
\includegraphics[scale=1.6]{images/AF_Logotype20141_Black.png}
\vspace{16pt}
\\ \large ÅF Embedded Systems
\end{center}

\newpage

\tableofcontents				% Insert table of contents

\newpage

\section{Introduction}

The \emph{Dragonfly} project is an internal competence enhancement project for ÅF employees. The goal is to combine technology, competence and experience from various engineering fields in order to construct a highly advanced quadrotor UAV system. The focus of the Flight Control Board development deals with low-level maneuvering of the aircraft, calculating motor command based on a feedback control system. Some of the major technologies deployed to attain this are control theory, electronics and software development.

\section{System description}

	\subsection{Flight dynamics}
	
	\subsection{Motors}

\section{Control}

	\subsection{Controller design}
	
	\subsection{Attitude control}
	
	\subsection{Estimation theory}

Reference to sample Figure \ref{fig:sampleimage}.

\begin{figure}[h]
    \centering
    \includegraphics[scale=0.4]{images/quad_rendered.jpg}
    \caption{Sample image}
    \label{fig:sampleimage}
\end{figure}

Sample equation below in (\ref{eq7}).

\begin{equation}
\min e^{x_1} + x_{1}^2 + x_{1}x_{2} + \mu (\dfrac{1}{2}x_1 + x_2 - 1)^2
\label{eq7}
\end{equation}

Sample reference to \cite{stenberg}.

\section{Hardware}

\subsection{STM32F3Discovery Board}
The STM32F3Discovery is an evaluation board provided by STMicroelectronics. It features an STM32F303VCT6 microprocessor based on the ARM Cortex-M4 core. For evaluation purposes, the board has been fitted with accelerometer, magnetometer and gyroscope sensors, an on-board ST-Link/V2 debugger/programmer, various LED:s, extension headers for all I/O pins and more.

\begin{itemize}
  \item CPU speed: $72$ MHz
  \item ROM: 256 kB Flash
  \item SRAM: 48 kB (40 kB available to user)
  \item I/O pins: LQFP100 pin package with pins attached to extension header
  \item On-board ST-LINK/V2 programming and debugging device
  \item Power supply: From USB bus or from an external 3 V or 5 V supply voltage
  \item L3GD20 MEMS gyroscope
  \item LSM303DLHC MEMS accelerometer and magnetometer
  \item 10 LEDs
  \item Two pushbuttons (User and Reset buttons)
  \item USB USER Mini-B connector
\end{itemize}

\subsection{Motors}
The motors of choice for the Dragonfly are of the type \emph{T-motor U3}, which are sensorless (no built-in Hall sensor) brushless motors, designed to deliver high performance and reliability. They are resistant to dirt and water. There are several benefits of using brushless motors instead of traditional DC motors. These include longer life-time (due to no brushes that become worn out), less noisy operation and more efficient cooling.

\begin{itemize}
  \item Motor velocity constant ($K_V$): $700$ $rpm/V$
  \item Configuration: 12N-14P (12 stator poles, 14 magnet poles) 
  \item Motor dimensions (d x l): $41.8$ $\times$ $30.75$ $mm$
  \item Weight: $97$ $g$
  \item ldle current at $10$ $V$: $0.5$ $A$
  \item No. of cells (LiPo): 3-4S
  \item Max continuous current ($180$ $s$): $25$ $A$
  \item Max continuous power ($180$ $s$): $400$ $W$
  \item Max efficiency current: $4-10$ $A$ $>$ $82$ \%
  \item Internal resistance: $50$ $m\Omega$
\end{itemize}

\section{Software}

\begin{thebibliography}{99}
\bibitem[1]{stenberg} Model-based Design Development and Control of a Wind Resistant Multirotor UAV, C. Månsson, D. Stenberg, Lunds Tekniska Högskola 2014
\end{thebibliography}

\end{document}                  % Slut p� dokumentet
